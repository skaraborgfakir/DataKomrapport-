%% Time-stamp: <2019-04-15 16:37:14 nsa>
%% 
\documentclass[swedish,11pt,a4paper]{article}

\usepackage[T1]{fontenc}
\usepackage[utf8]{inputenc}

%% \renewcommand{\baselinestretch}{1.5}
%% \linespread{1.5}

% smallare marginal för anteckningar
% \usepackage[marginparwidth=30pt]{geometry}

%% 
%% använd Times istället för Computer Modern
\usepackage{mathptmx}
%% \usepackage{mathpazo}

\usepackage{csquotes}       %% babel är för svenska beroende av csquotes
\usepackage[swedish]{babel} %% babel

\usepackage[bottom]{footmisc} %% fotnötter i sidans nederkant



\usepackage{verbatim} %% verbatim miljö
% \usepackage{hyperref}
% \usepackage{footnote}

\usepackage{xcolor} %% \definecolor, den gråa bakgrunden i kodlistningarna

\usepackage{listings} %% programkodslistningar
\lstset{basicstyle=\footnotesize,
  captionpos=b,
  backgroundcolor=\color{light-gray},
  extendedchars=true,
  breaklines=true,
  literate={ö}{{\"o}}1 {å}{{\aa}}1 {Å}{{\aA}}l {ä}{{\"a}}1 {Ö}{{\"O}}1 {Ä}{{\"A}}1
}


\usepackage{tabularx}                  %% mer användbara tabeller
\usepackage{tabu}                      %% mer användbara tabeller
\usepackage[flushleft]{threeparttable} %% fotnoter inuti tabellerna
\usepackage{booktabs}                  %% bottomrule/toprule för tabellerna

% \usepackage[fleqn]{mathtools}
% \usepackage{pgf}
% \newcommand{\BinaryAdd}[2]{%
% {#1}_2 + {#2}_2 = \pgfmathbin{0b#1 + 0b#2}% compute binary addition
% \pgfmathresult_2% format output to make it clear it is base 2
% }%


%%   \usepackage[style=alphabetic,backend=biber]{biblatex} %% alfabetisk stil [DKatz1990]
\usepackage[style=authoryear,backend=biber]{biblatex} %% 
\addbibresource[location=local]{bib.bib}

\title{Labbrapport IT119G}
\author{Stefan Niskanen Skoglund\\IT119G\\19670927--5934}

\begin{document}

\maketitle

\section{Statisk IP:adressering}\label{sec:statisk_ip_adressering}

Användning av ARP och IP:adressering i datornätverk.

\subsection{översättning av IP-adress till de motsvarande MAC:adresserna}\label{trans_ip_mac}

\begin{table}[ht]
  \caption{maskinnamn \& MAC-adresser}
  \begin{tabu}{ll}
    \toprule
    ESXi (vmnic0)     &6c:0b:84:08:fa:79\\
    PC01 (e160)       &00:50:56:36:16:67\footnote{Statisk tilldelning via gui}\\
    webserver01 (e160)&00:0c:29:4a:cc:d2\footnote{automatiskt tilldelad adress i ESXi}\\
    \bottomrule
  \end{tabu}
\end{table}

Vilket protokoll används i ett nätverk där datorer vill översätta IPv4:s logiska adresser till något 
som hårdvaran kan hantera? ARP = address resolution protocol.

Vad kan man säga om vem som är leverantör av en maskin om man studerar en MAC:adress i ARP:tabellen?

\begin{table}[h!]
  \centering
  \begin{tabu}[lll]
    \toprule
    maskinnamn         &MAC-adress         & Ansvarig leverantör\\
    \toprule
    ESXi (vmnic0)      & 6c:0b:84:08:fa:79 & Universal Global Scientific Industrial Co.,Ltd.\\
    PC01 (e160)        & 00:50:56:36:16:67 & VMware Inc\\
    webserver01 (e160) & 00:0c:29:4a:cc:d2 & VMware Inc\\
    dhcp01 (e160)      & 00:0c:29:9e:43:6a & VMware Inc\\
    10.204.20.1        & bc:ea:fa:11:94:33 & Hewlett-Packard\\
    10.204.8.1         & bc:ea:fa:11:94:47 & Hewlett-Packard\\
    \bottomrule
  \end{tabu}  
\end{table}

    % \begin{tablenotes}
    % \item [1] Lenovo PC
    % \item [2] Hewlett-Packard växel? Anslutning i gateway:maskin för gruppen 10.204.20.0
    % \item [3] Hewlett-Packard växel? Anslutning i gateway:maskin för gruppen 10.204.8.0
    % \item [4] Har använt mer än en datorgrupp
    % \end{tablenotes}


    MAC-adresser i klassisk Ethernet som även är känt som DIX:standarden efter
    de bakomliggande leverantörerna fördelar användbara adresser genom att tilldela
    leverantörerna ett eller flera prefix eller adressområden som de blev ansvarig för. MAC-adressen är 48 bitar lång
    och prefixet är hälften av det dvs 24 bitar. Leverantören använder
    de 24 övriga bitarna för att differentiera mellan maskiner men även anslutningskort för ethernet.

    Många prefix reserverades från första början för olika användningar. Ett exempel på
    sådan användning är multicast i IP:nätverk där adressen 224.20.20.2 nominerar
    de maskiner som är medlemmar av en viss grupp.

    Varför syns det inte någon förändring i ARP-tabellen efter att
    ha försökt ping:a specifik maskin bortanför utgående anslutning ?
    PC01 har i ARP:tabellen lagt till VMrouter:maskinens MAC-adress.

    \section{Trafikspårning}
    \label{sec:wireshark_usage}

    Vilket protokoll använder ping\footnote{se man -s 8 ping} ? ICMP:s echo:begäran (ICMP typ 8) och ICMP:s motsvarande svarspaket (ICMP typ 0.)

    Eftersom trafik till andra maskiner än de fyra som är direkt åtkomliga i ``Client network'' skickas via
    ``VMrouter'' behövs enbart den maskinens MAC:adress i ARP:tabellen (arp funktionen i operativsystemet hanterar
    next-hop:adresser.)\@10.204.20.1 är inte någon av dem.

    \section{DHCP}
    \label{sec:dhcp_konf}

    Jag flyttade ``pc01'' till  ``Server network'' sedan klagade den maskinen på ``jag är bortkopplad !''.
    Dessutom fick den inte någon ny IP:adress eftersom det inte finns någon ``DHCP'' tjänst i det nätet.

    Däremot sänds det på det nätet ut ett antal ``DHCP Discover''.

    Normalt sett skickas ``broadcast'':paket (mottagaradress 255.255.255.255 med avsändare 0.0.0.0) till port 67 från port 68 med UDP från
    den klient som vill ha en ny adress (``DHCP Request''). Dhcp:kontrollanten svarar med ett ``DHCP ACK'' från sin
    egen IPv4:adress till klientens MAC-adress med klientens korrekta IPv4:adress och andra optioner som exempelvis
    korrekt DNS:maskin och DNS:domännamn i samma paket.

    \newpage

    \subsection{loggar}

    ``/var/lib/dhcp/dhcpd.leases'' innehåller uppgifter om vilka klienter som har fått
    automatisk tilldelade adresser via ``range'':uppgiften. Om maskinen inte syns i leases och
    har en statiskt tilldelad adress så är det fullt normalt - dpcpd.leases har inte
    uppgifter om maskiner vars adresser är statiskt tilldelade
    som exempelvis för pc01:

    \definecolor{light-gray}{gray}{0.95}

    % \lstset{basicstyle=\footnotesize,
    % captionpos=b,
    % backgroundcolor=\color{light-gray},
    % extendedchars=true,
    % breaklines=true,
    % literate={ö}{{\"o}}1 {å}{{\aa}}1 {Å}{{\aA}}l {ä}{{\"a}}1 {Ö}{{\"O}}1 {Ä}{{\"A}}1
    % }
    %   \begin{lstlisting}[caption={/etc/dhcp/dhcpd.conf}]
    %     host  pc01 {
    %     hardware ethernet 00:50:56:36:16:67;
    %     fixed-address 192.168.1.11;
    %   }
    %   \end{lstlisting}

    %   \begin{lstlisting}[caption={/var/lib/dhcp/dhcpd.leases}]
    %     lease 192.168.1.11 {
    %     starts 5 2019/03/29 11:25:55;
    %     ends 5 2019/03/29 11:35:55;
    %     cltt 5 2019/03/29 11:25:55;
    %     binding state active;
    %     next binding state free;
    %     rewind binding state free;
    %     hardware ethernet 00:50:56:36:16:67;
    %     client-hostname "pc01";
    %   }
    %   \end{lstlisting}

    \section{broadcastdomäner}
    \label{sec:broadcastdomains}

    VMrouter borde vara ansluten till tre st (det lokala arbetsgruppsnätet, Server network och Client network.)

    \section{DNS:konfiguration}
    \label{sec:dns_konf}

    % \begin{lstlisting}[caption={Några frågor, dig +search www}]
    %   ; <<>> DiG 9.9.4-RedHat-9.9.4-73.el7_6 <<>> +search www
    %   ;; global options: +cmd
    %   ;; Got answer:
    %   ;; ->>HEADER<<- opcode: QUERY, status: NOERROR, id: 63149
    %   ;; flags: qr aa rd ra; QUERY: 1, ANSWER: 2, AUTHORITY: 1, ADDITIONAL: 2

    %   ;; OPT PSEUDOSECTION:
    %   ; EDNS: version: 0, flags:; udp: 4096
    %   ;; QUESTION SECTION:
    %   ;www.b18steni.it119g.nsa.his.se.	IN	A

    %   ;; ANSWER SECTION:
    %   www.b18steni.it119g.nsa.his.se.	10 IN	CNAME	webserver01.b18steni.it119g.nsa.his.se.
    %   webserver01.b18steni.it119g.nsa.his.se.	10 IN A	192.168.2.21

    %   ;; AUTHORITY SECTION:
    %   b18steni.it119g.nsa.his.se. 10	IN	NS	ns01.b18steni.it119g.nsa.his.se.

    %   ;; ADDITIONAL SECTION:
    %   ns01.b18steni.it119g.nsa.his.se. 10 IN	A	192.168.2.22

    %   ;; Query time: 0 msec
    %   ;; SERVER: 192.168.2.22#53(192.168.2.22)
    %   ;; WHEN: fre mar 29 13:32:30 CET 2019
    %   ;; MSG SIZE  rcvd: 136
    % \end{lstlisting}

    % \begin{lstlisting}[caption={Några frågor, dig -t NS b18steni.it119g.nsa.his.se.}]
    %   ; <<>> DiG 9.9.4-RedHat-9.9.4-73.el7_6 <<>> -t NS b18steni.it119g.nsa.his.se.
    %   ;; global options: +cmd
    %   ;; Got answer:
    %   ;; ->>HEADER<<- opcode: QUERY, status: NOERROR, id: 20764
    %   ;; flags: qr aa rd ra; QUERY: 1, ANSWER: 1, AUTHORITY: 0, ADDITIONAL: 2

    %   ;; OPT PSEUDOSECTION:
    %   ; EDNS: version: 0, flags:; udp: 4096
    %   ;; QUESTION SECTION:
    %   ;b18steni.it119g.nsa.his.se.	IN	NS

    %   ;; ANSWER SECTION:
    %   b18steni.it119g.nsa.his.se. 10	IN	NS	ns01.b18steni.it119g.nsa.his.se.

    %   ;; ADDITIONAL SECTION:
    %   ns01.b18steni.it119g.nsa.his.se. 10 IN	A	192.168.2.22

    %   ;; Query time: 0 msec
    %   ;; SERVER: 192.168.2.22#53(192.168.2.22)
    %   ;; WHEN: fre mar 29 13:34:03 CET 2019
    %   ;; MSG SIZE  rcvd: 90
    % \end{lstlisting}

    % \begin{lstlisting}[caption={Några frågor, dig -t A  ns01.b18steni.it119g.nsa.his.se.}]
    %   ;; global options: +cmd
    %   ;; Got answer:
    %   ;; ->>HEADER<<- opcode: QUERY, status: NOERROR, id: 38195
    %   ;; flags: qr aa rd ra; QUERY: 1, ANSWER: 1, AUTHORITY: 1, ADDITIONAL: 1

    %   ;; OPT PSEUDOSECTION:
    %   ; EDNS: version: 0, flags:; udp: 4096
    %   ;; QUESTION SECTION:
    %   ;ns01.b18steni.it119g.nsa.his.se. IN	A

    %   ;; ANSWER SECTION:
    %   ns01.b18steni.it119g.nsa.his.se. 10 IN	A	192.168.2.22

    %   ;; AUTHORITY SECTION:
    %   b18steni.it119g.nsa.his.se. 10	IN	NS	ns01.b18steni.it119g.nsa.his.se.

    %   ;; Query time: 0 msec
    %   ;; SERVER: 192.168.2.22#53(192.168.2.22)
    %   ;; WHEN: fre mar 29 13:35:29 CET 2019
    %   ;; MSG SIZE  rcvd: 90
    % \end{lstlisting}

    % \begin{lstlisting}[caption={Några frågor, dig -t SOA @ns1.google.com. google.com.}]
    %   ;; Got answer:
    %   ;; ->>HEADER<<- opcode: QUERY, status: NOERROR, id: 24539
    %   ;; flags: qr aa rd; QUERY: 1, ANSWER: 1, AUTHORITY: 4, ADDITIONAL: 9
    %   ;; WARNING: recursion requested but not available

    %   ;; OPT PSEUDOSECTION:
    %   ; EDNS: version: 0, flags:; udp: 512
    %   ;; QUESTION SECTION:
    %   ;google.com.			IN	SOA

    %   ;; ANSWER SECTION:
    %   google.com.		60	IN	SOA	ns1.google.com. dns-admin.google.com. 240952484 900 900 1800 60

    %   ;; AUTHORITY SECTION:
    %   google.com.		345600	IN	NS	ns2.google.com.
    %   google.com.		345600	IN	NS	ns4.google.com.
    %   google.com.		345600	IN	NS	ns1.google.com.
    %   google.com.		345600	IN	NS	ns3.google.com.

    %   ;; ADDITIONAL SECTION:
    %   ns2.google.com.		345600	IN	A	216.239.34.10
    %   ns2.google.com.		345600	IN	AAAA	2001:4860:4802:34::a
    %   ns4.google.com.		345600	IN	A	216.239.38.10
    %   ns4.google.com.		345600	IN	AAAA	2001:4860:4802:38::a
    %   ns1.google.com.		345600	IN	A	216.239.32.10
    %   ns1.google.com.		345600	IN	AAAA	2001:4860:4802:32::a
    %   ns3.google.com.		345600	IN	A	216.239.36.10
    %   ns3.google.com.		345600	IN	AAAA	2001:4860:4802:36::a

    %   ;; Query time: 25 msec
    %   ;; SERVER: 216.239.32.10#53(216.239.32.10)
    %   ;; WHEN: fre mar 29 13:37:10 CET 2019
    %   ;; MSG SIZE  rcvd: 333
    % \end{lstlisting}

    % \begin{lstlisting}[caption={Några frågor, 'dig -t A www' med eller utan '+search'}]
    %   dig +search www

    %   ; <<>> DiG 9.10.3-P4-Ubuntu <<>> +search www
    %   ;; global options: +cmd
    %   ;; Got answer:
    %   ;; ->>HEADER<<- opcode: QUERY, status: NOERROR, id: 61521
    %   ;; flags: qr aa rd ra; QUERY: 1, ANSWER: 2, AUTHORITY: 1, ADDITIONAL: 2

    %   ;; OPT PSEUDOSECTION:
    %   ; EDNS: version: 0, flags:; udp: 4096
    %   ;; QUESTION SECTION:
    %   ;www.b18steni.it119g.nsa.his.se.	IN	A

    %   ;; ANSWER SECTION:
    %   www.b18steni.it119g.nsa.his.se.	10 IN	CNAME	webserver01.b18steni.it119g.nsa.his.se.
    %   webserver01.b18steni.it119g.nsa.his.se.	10 IN A	192.168.2.21

    %   ;; AUTHORITY SECTION:
    %   b18steni.it119g.nsa.his.se. 10	IN	NS	ns01.b18steni.it119g.nsa.his.se.

    %   ;; ADDITIONAL SECTION:
    %   ns01.b18steni.it119g.nsa.his.se. 10 IN	A	192.168.2.22

    %   ;; Query time: 0 msec
    %   ;; SERVER: 192.168.2.22#53(192.168.2.22)
    %   ;; WHEN: Fri Mar 29 13:42:24 CET 2019
    %   ;; MSG SIZE  rcvd: 136

    %   dig www

    %   ; <<>> DiG 9.10.3-P4-Ubuntu <<>> www
    %   ;; global options: +cmd
    %   ;; Got answer:
    %   ;; ->>HEADER<<- opcode: QUERY, status: NXDOMAIN, id: 44523
    %   ;; flags: qr rd ra ad; QUERY: 1, ANSWER: 0, AUTHORITY: 1, ADDITIONAL: 1

    %   ;; OPT PSEUDOSECTION:
    %   ; EDNS: version: 0, flags:; udp: 4096
    %   ;; QUESTION SECTION:
    %   ;www.				IN	A

    %   ;; AUTHORITY SECTION:
    %   .			10692	IN	SOA	a.root-servers.net. nstld.verisign-grs.com. 2019032900 1800 900 604800 86400

    %   ;; Query time: 2 msec
    %   ;; SERVER: 192.168.2.22#53(192.168.2.22)
    %   ;; WHEN: Fri Mar 29 13:42:33 CET 2019
    %   ;; MSG SIZE  rcvd: 107
    % \end{lstlisting}

    \newpage
    \subsection{Betydelsen av punkt eller inte i frågor med dig}

    Dig:programmet kan använda den sökdomän i DNS som maskinerna är konfigurerade för, ``b18steni.it119g.nsa.his.se'' i mitt fall och ``xxxx.it119g.nsa.his.se''
    för övriga klasskamrater.
    Dock kräver ``dig'' att flaggan ``+search'' specificeras då eller att användaren självt specificerar domännamnet.

    De här frågorna är ekvivalenta:
\begin{verbatim}
dig www.
dig www
\end{verbatim}
    och dessa:
\begin{verbatim}
dig www.b18steni.it119g.nsa.his.se.
dig +search www
\end{verbatim}

    \section{DNS:konfiguration}\label{sec:dns_config}

    \section{WEB:tjänst}\label{sec:httpd_config}

    DNS i ns01 får frågan:
    Jag vill få en A:uppgift för maskinen ``webserver01.b18steni.it119g.nsa.his.se''.
    Har du någrar data om detta ?
    BIND:programmet svarar med motsvarande A:uppgift dvs vad ``webserver01.b18steni.it119g.nsa.his.se''
    heter och vilken IP:adress den har.

    ---
    ; <<>> DiG 9.10.3-P4-Ubuntu <<>> @192.168.2.22 webserver01.b18steni.it119g.nsa.his.se
    ; (1 server found)
    ;; global options: +cmd
    ;; Got answer:
    ;; ->>HEADER<<- opcode: QUERY, status: NOERROR, id: 9893
    ;; flags: qr aa rd; QUERY: 1, ANSWER: 1, AUTHORITY: 1, ADDITIONAL: 2
    ;; WARNING: recursion requested but not available

    ;; OPT PSEUDOSECTION:
    ; EDNS: version: 0, flags:; udp: 4096
    ;; QUESTION SECTION:
    ;webserver01.b18steni.it119g.nsa.his.se.	IN A

    ;; ANSWER SECTION:
    webserver01.b18steni.it119g.nsa.his.se.	10 IN A	192.168.2.21

    ;; AUTHORITY SECTION:
    b18steni.it119g.nsa.his.se. 10	IN	NS	ns01.b18steni.it119g.nsa.his.se.

    ;; ADDITIONAL SECTION:
    ns01.b18steni.it119g.nsa.his.se. 10 IN	A	192.168.2.22

    ;; Query time: 0 msec
    ;; SERVER: 192.168.2.22#53(192.168.2.22)
    ;; WHEN: Sun Apr 14 17:02:20 CEST 2019
    ;; MSG SIZE  rcvd: 118
    ---
    Det protokoll från applikationsnivån som används är DNS (eller domain.)
    Transportprotokollet för sådana här frågor är vanligtvis UDP med det
    undantaget att vissa frågor pga svaren blir för stora för ett UDP:paket
    sänds med TCP. Min DNS:tjänst i ns01 dvs 192.168.2.22.
    Avsändningsporten är 53683 och mottagande 53 (klienten.)
    Från klienten ser det ut som att svaret kommer från port 53 destinerat
    till 53683.

    \appendix

    I systemet modifierade filer inklusive DNS zoner.

    \section{DNS}\label{sec:appendix_bind_config}

    Egentligen konfiguration av bind\footnote{se ``man -s 8 bind''} självt.

    \begin{lstlisting}[caption={/etc/bind/named.conf.local}]
      //
      // avsedd för ns01.b18steni.it119g.nsa.his.se
      //
      // cfe-regler från 192.168.2.24
      // senast uppdatering av policy:
      //

      zone "b18steni.it119g.nsa.his.se" {
        type master;
        file "/etc/bind/zones/db.b18steni";
      };

      zone "168.192.in-addr.arpa" {
        type master;
        file "/etc/bind/zones/db.b18steni_192_rev";
      };
    \end{lstlisting}

    \begin{lstlisting}[caption={/etc/bind/named.conf.options}]
      //
      // avsedd för ns01.b18steni.it119g.nsa.his.se
      //
      // cfe-regler från 192.168.2.24
      // senast uppdatering av policy:
      //

      acl my_computers { 192.168.0.0/16; 10.204.12.0/24; 10.204.20.0/24; };

      options {
        directory "/var/cache/bind";
        dnssec-validation auto;

        forwarders {
          10.0.252.201;
          10.0.252.202;
        };

        listen-on-v6 { any; };
        auth-nxdomain no;    # conform to RFC1035

        allow-recursion { my_computers; };
        allow-query { any; };
      };
    \end{lstlisting}

    \section{DNS zoner}\label{sec:appendix_dns_zones}

    \texttt{db.b18steni.it119g.nsa.his.se} och motsvarande baklängesmapp\@
    \texttt{ db.192.168.2/db.192.168.1} för trädet \texttt{2.168.192.in-addr.arpa}
    och \texttt{1.168.192.in-addr.arpa}.

    \printbibliography{}
  \end{document}
